\documentclass[a4paper,12pt,twoside]{article}

\usepackage[a4paper, inner=2cm, outer=2cm,top=2cm, bottom=2cm]{geometry}
\usepackage[utf8]{inputenc}
\usepackage[portuges]{babel}
\usepackage{aeguill}
\usepackage{indentfirst}
\usepackage{xcolor}
\usepackage{graphicx}
\usepackage{titlesec}
\usepackage{titling}
\usepackage{wrapfig}
\usepackage{caption}
\usepackage{subcaption}
\usepackage{enumitem}
\usepackage{amsmath}
\usepackage{amssymb}
\usepackage{color}
\usepackage{setspace}
\usepackage{titlesec}
\usepackage{listings}

\input{/home/lucas/docs/latex/lstset}

\usepackage{fancyhdr} %pagenumber top right
\fancyhf{}
\fancyheadoffset{0cm}
\renewcommand{\headrulewidth}{0pt}
\renewcommand{\footrulewidth}{0pt}
\fancyhead[R]{\thepage}
\fancypagestyle{plain}{%
   \fancyhf{}%
   \fancyhead[R]{\thepage}%
}

\renewcommand*{\thepage}{\footnotesize\arabic{page}}


\titleformat{\section}
  {\normalfont\fontsize{12}{15}\bfseries}{\thesection}{1em}{}
\titleformat{\subsection}
  {\normalfont\fontsize{12}{15}\bfseries}{\thesubsection}{1em}{}
\titleformat{\subsubsection}
  {\normalfont\fontsize{12}{15}\bfseries}{\thesubsubsection}{1em}{}

\input{/home/lucas/docs/latex/hyper}

\hypersetup{
    pdftitle={Operação e Formação de preços},
    pdfkeywords={Operação} {Planejamento} {Preços} {Despacho},
    % pdftitle={My title},          % title
    % pdfsubject={Subject},         % subject of the document
    % pdfproducer={Producer},       % producer of the document
    % pdfkeywords={keyword1} {key2} % list of keywords
}

\newenvironment{boldenv}
{\bfseries}
{}
\onehalfspacing

\title{Sistema para Detecção de Infestação Parasitária em Pequenos Ruminantes}
\author{Lucas Budde Mior}

\begin{document}
\begin{titlepage}
   \large
   \begin{center}
       \begin{boldenv}
           UNIVERSIDADE FEDERAL DE SANTA CATARINA \\
           DEPARTAMENTO DE ENGENHARIA ELÉTRICA \\
           CURSO DE ENGENHARIA ELÉTRICA \\
           \vspace*{2cm}

           \vspace*{\fill}
           Operação e Formação de Preços \\
           \vspace*{1cm}
           Relatório \\
           \vspace*{1cm}
           Planejamento e Regulação de Mercados de Energia Elétrica
           \vspace*{\fill}

           Lucas Budde Mior \\
           \vspace{0.5cm}
           Professor: Erlon Finardi

           \vfill

           Florianópolis
       \end{boldenv}
   \end{center}
\end{titlepage}
\renewcommand{\contentsname}{Sumário}

\newgeometry{a4paper, inner=3cm, outer=2cm}
\pagestyle{empty}
\addtocontents{toc}{\protect\thispagestyle{empty}}
\tableofcontents
\newpage

\pagestyle{plain}
\section{Introdução} % revisão bibliográfica, justificativa e objetivos
Esse trabalho demonstra a simulação de um sistema elétrico alimentado por uma usina hidrelétrica e 3 térmicas, durante um período de 20 horas.
Para o despacho, é utilizado um modelo de otimização implementado em python utilizando a biblioteca Gurobipy.
Os cálculos de formação de preço também foram implemetados em python, com base no despacho otimizado.

\begin{figure}[h!]
    \centering
  \includegraphics[width=\linewidth]{sistema.png}
  \caption{Apresentação do sistema}\label{fig:sistema}
\end{figure}

\section{Questão 1 - Afluência Hidráulica e Demanda de Cada Barra}
O volume afluente é modelado como uma distribuição uniforme entre \(0\) e \(100hm³\).
Os valores sortados foram os seguintes:

\begin{center}
    \begin{tabular}{ c c c }
        Período & Volume \\
        0  & 77 \\
        1  & 28 \\
        2  & 17 \\
        3  & 41 \\
        4  & 97 \\
        5  & 18 \\
        6  & 13 \\
        7  & 21 \\
        8  & 95 \\
        9  & 0 \\
        10 & 62 \\
        11 & 4 \\
        12 & 30 \\
        13 & 58 \\
        14 & 23 \\
        15 & 87 \\
        16 & 67 \\
        17 & 60 \\
        18 & 47 \\
        19 & 69
    \end{tabular}
\end{center}
As demandas por barra, por sua vez, são modeladas como uma distribuição normal com média 25 e desvio padrão 2, com exceção da barra 1 que não possui carga.
As demandas por barra foram as seguintes:

\begin{center}
    \begin{tabular}{ c c c c }
        Período & L1 & L2 & L3 \\
        0  & 0 & 24 & 28 \\
        1  & 0 & 24 & 27 \\
        2  & 0 & 25 & 23 \\
        3  & 0 & 22 & 25 \\
        4  & 0 & 22 & 25 \\
        5  & 0 & 25 & 27 \\
        6  & 0 & 24 & 23 \\
        7  & 0 & 29 & 27 \\
        8  & 0 & 25 & 24 \\
        9  & 0 & 25 & 28 \\
        10 & 0 & 26 & 22 \\
        11 & 0 & 25 & 25 \\
        12 & 0 & 26 & 22 \\
        13 & 0 & 30 & 24 \\
        14 & 0 & 23 & 24 \\
        15 & 0 & 24 & 24 \\
        16 & 0 & 25 & 27 \\
        17 & 0 & 25 & 21 \\
        18 & 0 & 27 & 26 \\
        19 & 0 & 25 & 26
    \end{tabular}
\end{center}

\newpage
\section{Questão 2 - Despacho ótimo de cada usina e custo marginal de cada barra}

\newpage
\section{Questão 3 - Geração das Hidrelétricas}
\begin{figure}[h!]
    \centering
  \includegraphics[width=\linewidth]{foto.png}
  \caption{Geração das Hidrelétricas}\label{q3a}
\end{figure}

\begin{figure}[h!]
    \centering
  \includegraphics[width=\linewidth]{foto.png}
  \caption{Número de unidades despachadas}\label{q3b}
\end{figure}

\newpage
\section{Questão 4 - Rendimento Global e Produtibilidade}

\begin{figure}[h!]
    \centering
  \includegraphics[width=\linewidth]{foto.png}
  \caption{Rendimento Global}\label{q4a}
\end{figure}

\begin{figure}[h!]
    \centering
  \includegraphics[width=\linewidth]{foto.png}
  \caption{Produtibilidade}\label{q4b}
\end{figure}

\newpage
\section{Demanda Residual}
Aqui é observado o balanço de potência para cada estágio considerando apenas a geração eólica e hidrelétrica.
A demanda residual deverá ser atendida por termelétricas.

\begin{figure}[h!]
    \centering
  \includegraphics[width=\linewidth]{foto.png}
  \caption{Balanço de Potência}\label{q5}
\end{figure}

\newpage
\section{Questão 6 - Geração Térmica}

\begin{figure}[h!]
    \centering
  \includegraphics[width=0.7\linewidth]{foto.png}
  \caption{Informações das Usinas Termelétricas}\label{termicas}
\end{figure}

Considerando a tabela da figura \ref{termicas},
foi optado por priorizar o uso da térmica 1,
e evitar desligar a térmica 1 e 2.
Dessa forma pode-se evitar ao máximo o uso da usina mais cara (3).

Nota-se que em alguns momentos foi necessário reduzir a geração das térmicas 1 e 2,
para que no período seguinte, com menor demanda, fosse possível realizar o despacho.
Isso acarretou em um aumento no uso da térmica 3 e no custo total de operação.

Nota-se que, mesmo utilizando uma usina hipotética com geração e rampa ilimitadas,
ainda não é simples realizar esse despacho.
O custo total foi de R\$ \(1350260\).

\begin{figure}[h!]
    \centering
  \includegraphics[width=\linewidth]{foto.png}
  \caption{Geração Térmica}\label{q6}
\end{figure}

\newpage
\section{Questão 7 - Produtibilidade Constante}
Agora, a geração das hidrelétricas é modelada como constante.
O custo total foi de R\$ \(1336180\).
Ao analisar os resultados, não houve diferença significativa para essa simulação
a respeito do despacho das usinas termelétricas, no entanto,
num contexto maior, o resultado poderia ser comprometedor.

A operação inadequada das térmicas (muitas ligamentos e desligamentos,
uso de usinas mais caras) geraria um custo muito maior,
devido a pequenas imprecisões na modelagem das usinas hidrelétricas.
Dessa forma, esse modelo deve ser utilizado somente
no planejamento a longo prazo, não para a operação diária.

\begin{figure}[h!]
    \centering
  \includegraphics[width=0.7\linewidth]{foto.png}
  \caption{Produtibilidade constante para cada usina}\label{q7a}
\end{figure}

\begin{figure}[h!]
    \centering
  \includegraphics[width=\linewidth]{foto.png}
  \caption{Geração hidrelétrica para produtibilidade constante}\label{q7b}
\end{figure}

\begin{figure}[h!]
    \centering
  \includegraphics[width=\linewidth]{foto.png}
  \caption{Balanço de potência para produtibilidade constante}\label{q7c}
\end{figure}

\begin{figure}[h!]
    \centering
  \includegraphics[width=\linewidth]{foto.png}
  \caption{Geração térmica para produtibilidade constante das hidrelétricas}\label{q7d}
\end{figure}

\end{document}
