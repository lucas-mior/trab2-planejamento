\documentclass[a4paper,12pt,twoside]{article}

\usepackage[a4paper, inner=2cm, outer=2cm,top=2cm, bottom=2cm]{geometry}
\usepackage[utf8]{inputenc}
\usepackage[portuges]{babel}
\usepackage{aeguill}
\usepackage{indentfirst}
\usepackage{xcolor}
\usepackage{graphicx}
\usepackage{titlesec}
\usepackage{titling}
\usepackage{wrapfig}
\usepackage{caption}
\usepackage{subcaption}
\usepackage{enumitem}
\usepackage{amsmath}
\usepackage{amssymb}
\usepackage{color}
\usepackage{setspace}
\usepackage{titlesec}
\usepackage{listings}

\input{/home/lucas/docs/latex/lstset}

\usepackage{fancyhdr} %pagenumber top right
\fancyhf{}
\fancyheadoffset{0cm}
\renewcommand{\headrulewidth}{0pt}
\renewcommand{\footrulewidth}{0pt}
\fancyhead[R]{\thepage}
\fancypagestyle{plain}{%
   \fancyhf{}%
   \fancyhead[R]{\thepage}%
}

\renewcommand*{\thepage}{\footnotesize\arabic{page}}


\titleformat{\section}
  {\normalfont\fontsize{12}{15}\bfseries}{\thesection}{1em}{}
\titleformat{\subsection}
  {\normalfont\fontsize{12}{15}\bfseries}{\thesubsection}{1em}{}
\titleformat{\subsubsection}
  {\normalfont\fontsize{12}{15}\bfseries}{\thesubsubsection}{1em}{}

\input{/home/lucas/docs/latex/hyper}

\hypersetup{
    pdftitle={Operação e Formação de preços},
    pdfkeywords={Operação} {Planejamento} {Preços} {Despacho},
    % pdftitle={My title},          % title
    % pdfsubject={Subject},         % subject of the document
    % pdfproducer={Producer},       % producer of the document
    % pdfkeywords={keyword1} {key2} % list of keywords
}

\newenvironment{boldenv}
{\bfseries}
{}
\onehalfspacing

\title{Sistema para Detecção de Infestação Parasitária em Pequenos Ruminantes}
\author{Lucas Budde Mior}

\begin{document}
\begin{titlepage}
   \large
   \begin{center}
       \begin{boldenv}
           UNIVERSIDADE FEDERAL DE SANTA CATARINA \\
           DEPARTAMENTO DE ENGENHARIA ELÉTRICA \\
           CURSO DE ENGENHARIA ELÉTRICA \\
           \vspace*{2cm}

           \vspace*{\fill}
           Operação e Formação de Preços \\
           \vspace*{1cm}
           Relatório \\
           \vspace*{1cm}
           Planejamento e Regulação de Mercados de Energia Elétrica
           \vspace*{\fill}

           Lucas Budde Mior \\
           \vspace{0.5cm}
           Professor: Erlon Finardi

           \vfill

           Florianópolis
       \end{boldenv}
   \end{center}
\end{titlepage}
\renewcommand{\contentsname}{Sumário}

\newgeometry{a4paper, inner=3cm, outer=2cm}
\pagestyle{empty}
\addtocontents{toc}{\protect\thispagestyle{empty}}
\tableofcontents
\newpage

\pagestyle{plain}
\section{Introdução} % revisão bibliográfica, justificativa e objetivos
Esse trabalho demonstra a simulação de um sistema elétrico alimentado por uma usina hidrelétrica e 3 térmicas, durante um período de 20 horas.
Para o despacho, é utilizado um modelo de otimização implementado em python utilizando a biblioteca Gurobipy.
Os cálculos de formação de preço também foram implemetados em python, com base no despacho otimizado.

\begin{figure}[h!]
    \centering
  \includegraphics[width=\linewidth]{sistema.png}
  \caption{Apresentação do sistema}\label{fig:sistema}
\end{figure}

\section{Questão 1 - Afluência Hidráulica e Demanda de Cada Barra}
O volume afluente é modelado como uma distribuição uniforme entre \(0\) e \(100hm³\).
Os valores sortados foram os seguintes:

\begin{center}
    \begin{tabular}{ c c c }
        Período & Volume \\
        0  & 77 \\
        1  & 28 \\
        2  & 17 \\
        3  & 41 \\
        4  & 97 \\
        5  & 18 \\
        6  & 13 \\
        7  & 21 \\
        8  & 95 \\
        9  & 0 \\
        10 & 62 \\
        11 & 4 \\
        12 & 30 \\
        13 & 58 \\
        14 & 23 \\
        15 & 87 \\
        16 & 67 \\
        17 & 60 \\
        18 & 47 \\
        19 & 69
    \end{tabular}
\end{center}
As demandas por barra, por sua vez, são modeladas como uma distribuição normal com média 25 e desvio padrão 2, com exceção da barra 1 que não possui carga.
As demandas por barra foram as seguintes:

\begin{center}
    \begin{tabular}{ c c c c }
        Período & L1 & L2 & L3 \\
        0  & 0 & 24 & 28 \\
        1  & 0 & 24 & 27 \\
        2  & 0 & 25 & 23 \\
        3  & 0 & 22 & 25 \\
        4  & 0 & 22 & 25 \\
        5  & 0 & 25 & 27 \\
        6  & 0 & 24 & 23 \\
        7  & 0 & 29 & 27 \\
        8  & 0 & 25 & 24 \\
        9  & 0 & 25 & 28 \\
        10 & 0 & 26 & 22 \\
        11 & 0 & 25 & 25 \\
        12 & 0 & 26 & 22 \\
        13 & 0 & 30 & 24 \\
        14 & 0 & 23 & 24 \\
        15 & 0 & 24 & 24 \\
        16 & 0 & 25 & 27 \\
        17 & 0 & 25 & 21 \\
        18 & 0 & 27 & 26 \\
        19 & 0 & 25 & 26
    \end{tabular}
\end{center}

\newpage
\section{Questão 2 - Despacho ótimo de cada usina e custo marginal de cada barra}
O despacho ótimo obtido pelo modelo de otimização é apresentado a seguir (gerações em MW).

\begin{center}
    \begin{tabular}{ c c c c c c }
        Período & gt1 & gt2 & gt3 & gh & custo \\
        0   & 30.0  &  0.0  & 3.0  & 19.0  & R\$ 63.18 \\
        1   & 30.0  & 16.2  & 2.0  &  2.8  & R\$ 197.90 \\
        2   & 30.0  & 16.3  & 0.0  &  1.7  & R\$ 248.10 \\
        3   & 30.0  & 12.9  & 0.0  &  4.1  & R\$ 129.30 \\
        4   & 19.7  &  0.0  & 0.0  & 27.3  & R\$ 19.70 \\
        5   & 30.0  & 18.2  & 2.0  &  1.8  & R\$ 255.90 \\
        6   & 30.0  & 15.7  & 0.0  &  1.3  & R\$ 270.90 \\
        7   & 30.0  & 20.0  & 3.9  &  2.1  & R\$ 251.00 \\
        8   & 23.5  &  0.0  & 0.0  & 25.5  & R\$ 23.50 \\
        9   & 30.0  & 20.0  & 3.0  &  0.0  & R\$ 372.50 \\
        10  & 30.0  &  2.2  & 0.0  & 15.8  & R\$ 71.90 \\
        11  & 30.0  & 19.6  & 0.0  &  0.4  & R\$ 332.70 \\
        12  & 30.0  & 15.0  & 0.0  &  3.0  & R\$ 177.50 \\
        13  & 30.0  & 11.8  & 0.0  & 12.2  & R\$ 91.10 \\
        14  & 30.0  & 14.7  & 0.0  &  2.3  & R\$ 208.90 \\
        15  & 29.7  &  0.0  & 0.0  & 18.3  & R\$ 29.70 \\
        16  & 30.0  &  0.0  & 2.0  & 20.0  & R\$ 76.93 \\
        17  & 30.0  &  2.0  & 0.0  & 14.0  & R\$ 71.50 \\
        18  & 30.0  & 17.3  & 1.0  &  4.7  & R\$ 119.10 \\
        19  & 30.0  &  0.0  & 1.0  & 20.0  & R\$ 68.56
    \end{tabular}
\end{center}

\begin{itemize}
    \item{\textbf{gt1} - Geração na Usina Termelétrica 1}
    \item{\textbf{gt2} - Geração na Usina Termelétrica 2}
    \item{\textbf{gt3} - Geração na Usina Termelétrica 3}
    \item{\textbf{gh} - Geração na Usina Hidrelétrica}
\end{itemize}
Podemos observar que o otimizador limitou o uso da térmica 3, que praticamente só foi utilizada
em períodos de baixa afluência, e priorizou o uso da térmica 1 em relação a 2,
além de, obviamente, buscar utilizar a hidrelétrica o quanto possível.

Em seguida foi realizado os cálculo dos custos marginais de operação de cada barra.
Para isso utilizou-se o modelo de otimização e incrementou-se a demanda em 1MWh em cada barra.
A partir disso calcula-se o excedente de mercado (EM), considerando o CMO de cada barra, e o excedente de mercado devido a transmissão (EMT), considerando o maior dos CMOs para o período.
Na tabela a seguir pode-se visualizar os resultados.

\begin{center}
    \begin{tabular}{ c c c c c c c c c c }
        periodo & alfa   & f12   & f13   & f32  & cmo1  & cmo2  & cmo3        & EM     & EMT \\
        0    & 18.18   & 4.0  & 15.0 & -10.0  & 1.88  & 1.88  & 5.00  & 196.81   & 68.64 \\
        1   & 125.50 & -12.2  & 15.0 & -10.0  & 2.00  & 2.00  & 5.00   & 57.10   & 14.40 \\
        2   & 185.50 & -11.3  & 13.0 & -10.0  & 2.00  & 2.00  & 2.00 & -152.10    & 0.00 \\
        3    & 73.50 & -10.9  & 15.0 & -10.0  & 2.00  & 2.00  & 5.00  & 105.70   & 12.30 \\
        4     & 0.00  & 12.3  & 15.0 & -10.0  & 1.00  & 1.00  & 5.00  & 215.30  & 109.20 \\
        5   & 179.50 & -13.2  & 15.0 & -10.0  & 2.00  & 2.00  & 5.00    & 4.10   & 11.40 \\
        6   & 209.50 & -11.7  & 13.0 & -10.0  & 2.00  & 2.00  & 2.00 & -176.90    & 0.00 \\
        7   & 161.50 & -12.9  & 15.0  & -8.1  & 5.00  & 5.00  & 5.00   & 29.00    & 0.00 \\
        8     & 0.00  & 10.5  & 15.0  & -9.0  & 1.00  & 1.00  & 1.00   & 25.50    & 0.00 \\
        9   & 287.50 & -15.0  & 15.0 & -10.0  & 5.00  & 5.00  & 5.00 & -107.50    & 0.00 \\
        10   & 37.50   & 3.8  & 12.0 & -10.0  & 2.00  & 2.00  & 2.00   & 24.10    & 0.00 \\
        11  & 263.50 & -14.6  & 15.0 & -10.0  & 3.80  & 3.80  & 5.00  & -82.70    & 0.48 \\
        12  & 117.50  & -9.0  & 12.0 & -10.0  & 2.00  & 2.00  & 2.00  & -81.50    & 0.00 \\
        13   & 37.50  & -1.8  & 14.0 & -10.0  & 2.00  & 2.00  & 2.00   & 16.90    & 0.00 \\
        14  & 149.50 & -11.7  & 14.0 & -10.0  & 2.00  & 2.00  & 2.00 & -114.90    & 0.00 \\
        15    & 0.00   & 4.3  & 14.0 & -10.0  & 1.61  & 1.61  & 1.61   & 47.58    & 0.00 \\
        16   & 36.93   & 5.0  & 15.0 & -10.0  & 1.96  & 1.96  & 5.00  & 183.06   & 66.88 \\
        17   & 37.50   & 3.0  & 11.0 & -10.0  & 2.00  & 2.00  & 2.00   & 20.50    & 0.00 \\
        18   & 49.50 & -10.3  & 15.0 & -10.0  & 2.00  & 2.00  & 5.00  & 145.90   & 17.10 \\
        19   & 33.56   & 5.0  & 15.0 & -10.0  & 1.88  & 1.88  & 5.00  & 186.43   & 65.52
    \end{tabular}
\end{center}


\newpage

\section{Questão 3 e 4 - Contabilização no Mercado de Curto Prazo}
\subsection{Sem contrato}
A contabilização com ausência de contrato é apresentada a seguir:

\begin{center}
    \begin{tabular}{ c c c c c c c c }
        Período & Térmica 1  & Térmica 2  & Térmica 3  & Hidrelétrica  & Demanda 2  & Demanda 3      & EMT \\
        0    & 56.40     & 0.00     & 15.0  & 26.55   & 45.12  & 140.00  & -87.16 \\
        1    & 60.00    & 32.40     & 10.0   & 0.00   & 48.00  & 135.00  & -80.60 \\
        2    & 60.00    & 32.60      & 0.0   & 0.00   & 50.00   & 46.00   & -3.40 \\
        3    & 60.00    & 25.80      & 0.0   & 0.00   & 44.00  & 125.00  & -83.20 \\
        4    & 19.70     & 0.00      & 0.0  & 22.00   & 22.00  & 125.00 & -105.30 \\
        5    & 60.00    & 36.40     & 10.0   & 0.00   & 50.00  & 135.00  & -78.60 \\
        6    & 60.00    & 31.40      & 0.0   & 0.00   & 48.00   & 46.00   & -2.60 \\
        7   & 150.00   & 100.00     & 19.5   & 0.00  & 145.00  & 135.00  & -10.50 \\
        8    & 23.50     & 0.00      & 0.0  & 20.00   & 25.00   & 24.00   & -5.50 \\
        9   & 150.00   & 100.00     & 15.0   & 0.00  & 125.00  & 140.00    & 0.00 \\
        10   & 60.00     & 4.40      & 0.0  & 24.00   & 52.00   & 44.00   & -7.60 \\
        11  & 114.00    & 74.48      & 0.0   & 0.00   & 95.00  & 125.00  & -31.52 \\
        12   & 60.00    & 30.00      & 0.0   & 0.00   & 52.00   & 44.00   & -6.00 \\
        13   & 60.00    & 23.60      & 0.0  & 16.00   & 60.00   & 48.00   & -8.40 \\
        14   & 60.00    & 29.40      & 0.0   & 0.00   & 46.00   & 48.00   & -4.60 \\
        15   & 47.81     & 0.00      & 0.0  & 19.32   & 38.64   & 38.64  & -10.14 \\
        16   & 58.80     & 0.00     & 10.0  & 32.58   & 49.00  & 135.00  & -82.61 \\
        17   & 60.00     & 4.00      & 0.0  & 20.00   & 50.00   & 42.00   & -8.00 \\
        18   & 60.00    & 34.60      & 5.0   & 0.00   & 54.00  & 130.00  & -84.40 \\
        19   & 56.40     & 0.00      & 5.0  & 30.78   & 47.00  & 130.00  & -84.81
    \end{tabular}
\end{center}

\subsection{Com contrato}
O contrato utilizado foi o seguinte:
\begin{itemize}
    \item{Barra 2 - Contratou 15 MWh de \(T_1\) e 10 MWh de H}
    \item{Barra 3 - Contratou 5 MWh de \(T_1\), 10 MWh de \(T_2\), 5 MWh de \(T_3\) e 5 MWh de H}
\end{itemize}

A contabilização com contrato é apresentada a seguir:
\begin{center}
    \begin{tabular}{ c c c c c c c c }
        Período & Térmica 1  & Térmica 2  & Térmica 3  & Hidrelétrica  & Demanda 2  & Demanda 3 & EMT \\
        0    & 18.80   & -18.80   & -10.00  & -11.04   & -1.88   & 15.00 & -34.16 \\
        1    & 20.00    & 12.40   & -15.00  & -40.00   & -2.00   & 10.00 & -30.60 \\
        2    & 20.00    & 12.60   & -10.00  & -40.00    & 0.00   & -4.00 & -13.40 \\
        3    & 20.00     & 5.80   & -25.00  & -40.00   & -6.00    & 0.00 & -33.20 \\
        4    & -0.30   & -10.00   & -25.00    & 2.00   & -3.00    & 0.00 & -30.30 \\
        5    & 20.00    & 16.40   & -15.00  & -40.00    & 0.00   & 10.00 & -28.60 \\
        6    & 20.00    & 11.40   & -10.00  & -40.00   & -2.00   & -4.00 & -12.60 \\
        7    & 50.00    & 50.00    & -5.50 & -100.00   & 20.00   & 10.00 & -35.50 \\
        8     & 3.50   & -10.00    & -5.00    & 0.00    & 0.00   & -1.00 & -10.50 \\
        9    & 50.00    & 50.00   & -10.00 & -100.00    & 0.00   & 15.00 & -25.00 \\
        10   & 20.00   & -15.60   & -10.00  & -16.00    & 2.00   & -6.00 & -17.60 \\
        11   & 38.00    & 36.48   & -25.00  & -76.00    & 0.00    & 0.00 & -26.52 \\
        12   & 20.00    & 10.00   & -10.00  & -40.00    & 2.00   & -6.00 & -16.00 \\
        13   & 20.00     & 3.60   & -10.00  & -24.00   & 10.00   & -2.00 & -18.40 \\
        14   & 20.00     & 9.40   & -10.00  & -40.00   & -4.00   & -2.00 & -14.60 \\
        15   & 15.60   & -16.10    & -8.05  & -12.88   & -1.61   & -1.61 & -18.19 \\
        16   & 19.60   & -19.60   & -15.00   & -6.61    & 0.00   & 10.00 & -31.61 \\
        17   & 20.00   & -16.00   & -10.00  & -20.00    & 0.00   & -8.00 & -18.00 \\
        18   & 20.00    & 14.60   & -20.00  & -40.00    & 4.00    & 5.00 & -34.40 \\
        19   & 18.80   & -18.80   & -20.00   & -6.81    & 0.00    & 5.00 & -31.81
    \end{tabular}
\end{center}

Os contratos reduziram a quantidade de compra de energia
no mercado de curto prazo, mas ocorrem pequenas diferenças
em relação a energia contratada e a efetivamente gerada e consumida,
 a serem sanadas de acordo com a regulamentação do referido mercado.

\end{document}
